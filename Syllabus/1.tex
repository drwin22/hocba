\documentclass[]{article}
\usepackage[margin=.5in]{geometry}
\newcommand*{\authorfont}{\fontfamily{phv}\selectfont}
\usepackage{lmodern}
\usepackage{abstract}
\renewcommand{\abstractname}{}    % clear the title
\renewcommand{\absnamepos}{empty} % originally center
\newcommand{\blankline}{\quad\pagebreak[2]}

\providecommand{\tightlist}{%
  \setlength{\itemsep}{0pt}\setlength{\parskip}{0pt}} 
\usepackage{longtable,booktabs}

\usepackage{parskip}
\usepackage{titlesec}
\titlespacing\section{0pt}{12pt plus 4pt minus 2pt}{6pt plus 2pt minus 2pt}
\titlespacing\subsection{0pt}{12pt plus 4pt minus 2pt}{6pt plus 2pt minus 2pt}

\titleformat*{\subsubsection}{\normalsize\itshape}

\usepackage{titling}
\setlength{\droptitle}{-.25cm}

%\setlength{\parindent}{0pt}
%\setlength{\parskip}{6pt plus 2pt minus 1pt}
%\setlength{\emergencystretch}{3em}  % prevent overfull lines 

\usepackage[T1]{fontenc}
\usepackage[utf8]{inputenc}

\usepackage{fancyhdr}
\pagestyle{fancy}
\usepackage{lastpage}
\renewcommand{\headrulewidth}{0.3pt}
\renewcommand{\footrulewidth}{0.0pt} 
\lhead{}
\chead{}
\rhead{\footnotesize Math 0910: Introduction Algebra -- Spring 2023}
\lfoot{}
\cfoot{\small \thepage/\pageref*{LastPage}}
\rfoot{}

\fancypagestyle{firststyle}
{
\renewcommand{\headrulewidth}{0pt}%
   \fancyhf{}
   \fancyfoot[C]{\small \thepage/\pageref*{LastPage}}
}

%\def\labelitemi{--}
%\usepackage{enumitem}
%\setitemize[0]{leftmargin=25pt}
%\setenumerate[0]{leftmargin=25pt}




\makeatletter
\@ifpackageloaded{hyperref}{}{%
\ifxetex
  \usepackage[setpagesize=false, % page size defined by xetex
              unicode=false, % unicode breaks when used with xetex
              xetex]{hyperref}
\else
  \usepackage[unicode=true]{hyperref}
\fi
}
\@ifpackageloaded{color}{
    \PassOptionsToPackage{usenames,dvipsnames}{color}
}{%
    \usepackage[usenames,dvipsnames]{color}
}
\makeatother
\hypersetup{breaklinks=true,
            bookmarks=true,
            pdfauthor={ ()},
             pdfkeywords = {},  
            pdftitle={Math 0910: Introduction Algebra},
            colorlinks=true,
            citecolor=blue,
            urlcolor=blue,
            linkcolor=magenta,
            pdfborder={0 0 0}}
\urlstyle{same}  % don't use monospace font for urls


\setcounter{secnumdepth}{0}

\usepackage{color}
\usepackage{fancyvrb}
\newcommand{\VerbBar}{|}
\newcommand{\VERB}{\Verb[commandchars=\\\{\}]}
\DefineVerbatimEnvironment{Highlighting}{Verbatim}{commandchars=\\\{\}}
% Add ',fontsize=\small' for more characters per line
\usepackage{framed}
\definecolor{shadecolor}{RGB}{248,248,248}
\newenvironment{Shaded}{\begin{snugshade}}{\end{snugshade}}
\newcommand{\AlertTok}[1]{\textcolor[rgb]{0.94,0.16,0.16}{#1}}
\newcommand{\AnnotationTok}[1]{\textcolor[rgb]{0.56,0.35,0.01}{\textbf{\textit{#1}}}}
\newcommand{\AttributeTok}[1]{\textcolor[rgb]{0.77,0.63,0.00}{#1}}
\newcommand{\BaseNTok}[1]{\textcolor[rgb]{0.00,0.00,0.81}{#1}}
\newcommand{\BuiltInTok}[1]{#1}
\newcommand{\CharTok}[1]{\textcolor[rgb]{0.31,0.60,0.02}{#1}}
\newcommand{\CommentTok}[1]{\textcolor[rgb]{0.56,0.35,0.01}{\textit{#1}}}
\newcommand{\CommentVarTok}[1]{\textcolor[rgb]{0.56,0.35,0.01}{\textbf{\textit{#1}}}}
\newcommand{\ConstantTok}[1]{\textcolor[rgb]{0.00,0.00,0.00}{#1}}
\newcommand{\ControlFlowTok}[1]{\textcolor[rgb]{0.13,0.29,0.53}{\textbf{#1}}}
\newcommand{\DataTypeTok}[1]{\textcolor[rgb]{0.13,0.29,0.53}{#1}}
\newcommand{\DecValTok}[1]{\textcolor[rgb]{0.00,0.00,0.81}{#1}}
\newcommand{\DocumentationTok}[1]{\textcolor[rgb]{0.56,0.35,0.01}{\textbf{\textit{#1}}}}
\newcommand{\ErrorTok}[1]{\textcolor[rgb]{0.64,0.00,0.00}{\textbf{#1}}}
\newcommand{\ExtensionTok}[1]{#1}
\newcommand{\FloatTok}[1]{\textcolor[rgb]{0.00,0.00,0.81}{#1}}
\newcommand{\FunctionTok}[1]{\textcolor[rgb]{0.00,0.00,0.00}{#1}}
\newcommand{\ImportTok}[1]{#1}
\newcommand{\InformationTok}[1]{\textcolor[rgb]{0.56,0.35,0.01}{\textbf{\textit{#1}}}}
\newcommand{\KeywordTok}[1]{\textcolor[rgb]{0.13,0.29,0.53}{\textbf{#1}}}
\newcommand{\NormalTok}[1]{#1}
\newcommand{\OperatorTok}[1]{\textcolor[rgb]{0.81,0.36,0.00}{\textbf{#1}}}
\newcommand{\OtherTok}[1]{\textcolor[rgb]{0.56,0.35,0.01}{#1}}
\newcommand{\PreprocessorTok}[1]{\textcolor[rgb]{0.56,0.35,0.01}{\textit{#1}}}
\newcommand{\RegionMarkerTok}[1]{#1}
\newcommand{\SpecialCharTok}[1]{\textcolor[rgb]{0.00,0.00,0.00}{#1}}
\newcommand{\SpecialStringTok}[1]{\textcolor[rgb]{0.31,0.60,0.02}{#1}}
\newcommand{\StringTok}[1]{\textcolor[rgb]{0.31,0.60,0.02}{#1}}
\newcommand{\VariableTok}[1]{\textcolor[rgb]{0.00,0.00,0.00}{#1}}
\newcommand{\VerbatimStringTok}[1]{\textcolor[rgb]{0.31,0.60,0.02}{#1}}
\newcommand{\WarningTok}[1]{\textcolor[rgb]{0.56,0.35,0.01}{\textbf{\textit{#1}}}}

\usepackage{graphicx}
% We will generate all images so they have a width \maxwidth. This means
% that they will get their normal width if they fit onto the page, but
% are scaled down if they would overflow the margins.
\makeatletter
\def\maxwidth{\ifdim\Gin@nat@width>\linewidth\linewidth
\else\Gin@nat@width\fi}
\makeatother
\let\Oldincludegraphics\includegraphics
\renewcommand{\includegraphics}[1]{\Oldincludegraphics[width=\maxwidth]{#1}}



\usepackage{setspace}

\title{Math 0910: Introduction Algebra}
\author{Prof.~Francois Nguyen}
\date{Spring 2023}


\begin{document}  

		\maketitle
		
	
		\thispagestyle{firststyle}

%	\thispagestyle{empty}


	\noindent \begin{tabular*}{\textwidth}{ @{\extracolsep{\fill}} lr @{\extracolsep{\fill}}}


E-mail: \texttt{\href{mailto:Francois.nguyen@usaintpaul.edu}{\nolinkurl{Francois.nguyen@usaintpaul.edu}}} & Web: \href{http://MymathLab.com}{\tt MymathLab.com}\\
Office Hours: TTh 9:30 to 11:00 am and by appointment  &  Class Hours: T
11:00 am to 1:50 pm\\
Office: 324 Schaeffer Hall  & Class Room: 4140\\
	&  \\
	\hline
	\end{tabular*}
	
\vspace{2mm}
	


\begin{Shaded}
\begin{Highlighting}[]
\NormalTok{c }\OtherTok{\textless{}{-}} \FunctionTok{read.csv}\NormalTok{(}\StringTok{\textquotesingle{}attend{-}grade{-}relationships.csv\textquotesingle{}}\NormalTok{)}

\FunctionTok{plot}\NormalTok{(c}\SpecialCharTok{$}\NormalTok{attendance,c}\SpecialCharTok{$}\NormalTok{grade)}
\end{Highlighting}
\end{Shaded}

\includegraphics{1_files/figure-latex/unnamed-chunk-3-1.pdf}

\begin{Shaded}
\begin{Highlighting}[]
\NormalTok{firstday }\OtherTok{\textless{}{-}} \StringTok{"2023{-}01{-}09"}
    
\NormalTok{meetings }\OtherTok{\textless{}{-}} \FunctionTok{ymd}\NormalTok{(firstday) }\SpecialCharTok{+} \FunctionTok{c}\NormalTok{(}\DecValTok{0}\SpecialCharTok{:}\DecValTok{15}\NormalTok{) }\SpecialCharTok{*} \FunctionTok{weeks}\NormalTok{(}\DecValTok{1}\NormalTok{)}

\NormalTok{meeting\_headers }\OtherTok{\textless{}{-}} \FunctionTok{paste0}\NormalTok{(}\StringTok{"Week "}\NormalTok{, }\DecValTok{1}\SpecialCharTok{:}\DecValTok{16}\NormalTok{, }\StringTok{", "}\NormalTok{, }\FunctionTok{months}\NormalTok{(meetings), }\StringTok{" "}\NormalTok{, }\FunctionTok{day}\NormalTok{(meetings))}
\end{Highlighting}
\end{Shaded}

\hypertarget{overview}{%
\section{Overview:}\label{overview}}

This course is Introduction to Algebra Online course. The topics include
real numbers, methods of solving equations and inequalities and their
applications, exponents and polynomials, factoring polynomials, solving
quadratic equations and their applications, rational expressions,
rational exponents and radicals, and graphing functions (linear and
quadratic).

\hypertarget{course-requirements}{%
\section{Course Requirements}\label{course-requirements}}

A minimum grade of C is required in this course to progress to COURSE.
\# Course Objectives:

At the completion of this course, students will be able to:

\begin{itemize}
\tightlist
\item
  Identify and understand the basic concepts of algebraic expressions.
\item
  Perform operations on polynomial and rational expressions.
\item
  Apply the definition of absolute value to solve inequalities and
  equations involving absolute values.
\item
  Find solutions to and graph systems of linear equations and
  inequalities.
\item
  Solve equations involving first and second-degree polynomials and
  rational expressions.
\item
  Manipulate radical expressions using laws of exponents.
\item
  Understand basic properties of functions.
\item
  Apply properties of rational and radical expressions, polynomials, and
  absolute value in the context of real-life
\end{itemize}

\hypertarget{course-policies}{%
\section{Course Policies:}\label{course-policies}}

\hypertarget{general}{%
\subsection{General:}\label{general}}

-- Computers, Calculators are not to be used unless instructed to do so.
-- Quizzes and exams are closed book, closed notes.

\hypertarget{grade}{%
\subsection{Grade:}\label{grade}}

makeup quizzes or exams will be given.

-- Grades in the C range represent performance that meets expectations;
Grades in the B range represent performance that is substantially better
than the expectations; Grades in the A range represent work that is
excellent.

-- Grades will be maintained in the LMS course shell. Students are
responsible for tracking their progress by referring to the online
gradebook.

\hypertarget{assignments}{%
\section{Assignments}\label{assignments}}

-- Students are expected to work independently. Offering and accepting
solutions from others is an act of plagiarism, which is a serious
offense and all involved parties will be penalized according to the
Academic Honesty Policy.

-- Discussion among students is encouraged, but when in doubt, direct
your questions to the professor, tutor, or lab assistant.

-- No late assignments will be accepted under any circumstances.

\hypertarget{attendance-and-absences}{%
\section{Attendance and Absences}\label{attendance-and-absences}}

-- Attendance is expected and will be taken each class. You are allowed
to miss 1 class during the semester without penalty. Any further
absences will result in point and/or grade deductions.

-- Students are responsible for all missed work, regardless of the
reason for absence. It is also the absentee's responsibility to get all
missing notes or materials.

\hypertarget{academic-honesty-policy-summary}{%
\section{Academic Honesty Policy
Summary:}\label{academic-honesty-policy-summary}}

In addition to skills and knowledge, Saint Paul College aims to teach
students appropriate Ethical and Professional Standards of Conduct. The
Academic Honesty Policy exists to inform students and Faculty of their
obligations in upholding the highest standards of professional and
ethical integrity. All student work is subject to the Academic Honesty
Policy. Professional and Academic practice provides guidance about how
to properly cite, reference, and attribute the intellectual property of
others. Any attempt to deceive a faculty member or to help another
student to do so will be considered a violation of this standard.

\hypertarget{college-policy}{%
\section{College Policy}\label{college-policy}}

The College's academic honesty policies can be found in the catalog.
Please be assured that there will be absolutely no tolerance for
cheating in any way. All your quizzes and exams must be done
independently with no help from anyone. Make sure you copy the code of
honor statement below and send me an email with your name and date of
the time to confirm you adhere to this course policy: ``I will register
for only one account on Mymathlab.com. My answers to homework, quizzes
and exams will be my own work (except for assignments that explicitly
permit collaboration).

I will not make solutions to homework, quizzes or exams available to
anyone else. This includes both solutions written by me, as well as any
official solutions provided by the course instructor. I will not engage
in any other activities that will dishonestly improve my results or
dishonestly improve/hurt the results of others. Your name. Dated on
Jan/\ldots/2023.

\hypertarget{special-services-and-math-tutoring}{%
\section{Special Services and Math
Tutoring}\label{special-services-and-math-tutoring}}

Special Accommodations: It is college policy to provide reasonable
accommodations to students with disabilities. Please contact the office
of Disability Resources and Access if you wish to discuss this policy.
The Math Center (MC) will be open for drop-in tutoring in room 3125.
This tutoring is FREE and no appointment is necessary.

\hypertarget{online-coursework-integrity-declaration}{%
\section{Online coursework Integrity
Declaration}\label{online-coursework-integrity-declaration}}

Online submission of, or placing one's name on an exam, assignment, or
any course document is a statement of academic honor that the student
has not received or given inappropriate assistance in completing it and
that the student has complied with the Academic Honesty Policy in that
work.

\hypertarget{consequences}{%
\section{Consequences}\label{consequences}}

An instructor may impose a sanction on the student that varies depending
upon the instructor's evaluation of the nature and gravity of the
offense. Possible sanctions include but are not limited to, the
following: (1) Require the student to redo the assignment; (2) Require
the student to complete another assignment; (3) Assign a grade of zero
to the assignment; (4) Assign a final grade of ``F'' for the course. A
student may appeal these decisions according to the Academic Grievance
Procedure. (See the relevant section in the Student Handbook.) Multiple
violations of this policy will result in a referral to the Conduct
Review Board for possible additional sanctions.

The full text of the Academic Honesty Policy is in the Student Handbook.

\hypertarget{course-schedule}{%
\section{Course Schedule}\label{course-schedule}}

\hypertarget{week-1-january-9-introduction-to-the-course}{%
\subsection{Week 1, January 9: Introduction to the
Course}\label{week-1-january-9-introduction-to-the-course}}

\emph{Assignment}:

\begin{enumerate}
\def\labelenumi{\arabic{enumi}.}
\tightlist
\item
  \href{http://happygitwithr.com}{Bryan, Jennifer. 2016.
  \textit{Happy Git and GitHub for the UseR.} Chapters 1-16.} Read this
  carefully and follow its instructions to get set up with \textsf{R},
  RStudio, Git, and GitHub on your laptop before we meet.
\end{enumerate}

\hypertarget{week-2-january-16-the-replication-crisis-and-reproducibility}{%
\subsection{Week 2, January 16: The Replication Crisis and
Reproducibility}\label{week-2-january-16-the-replication-crisis-and-reproducibility}}

\emph{Readings}:

\begin{enumerate}
\def\labelenumi{\arabic{enumi}.}
\tightlist
\item
  \href{http://fivethirtyeight.com/features/how-two-grad-students-uncovered-michael-lacour-fraud-and-a-way-to-change-opinions-on-transgender-rights/}{Aschwanden,
  Christie, and Maggie Koerth-Baker. 2016. ``How Two Grad Students
  Uncovered an Apparent Fraud---and a Way to Change Opinions on
  Transgender Rights.'' \emph{FiveThirtyEight}, April 7}, and
  \href{http://blogs.lse.ac.uk/impactofsocialsciences/2013/04/24/reinhart-rogoff-revisited-why-we-need-open-data-in-economics/}{Dimitrova,
  Velichka. 2013. ``Reinhart-Rogoff Revisited: Coding Errors
  Happen---Key Problem Was in Not Making the Data Openly Available from
  the Start.'' \emph{LSE: The Impact Blog}, April 24.}
\item
  \href{http://journals.cambridge.org/action/displayAbstract?fromPage=online\&aid=9911378\&fulltextType=LT\&fileId=S2049847015000448}{Data
  Access and Research Transparency (DA-RT): A Joint Statement by
  Political Science Journal Editors.}
\item
  \href{https://ajps.org/ajps-replication-policy/}{\emph{AJPS}
  Replication and Verification Policy} and
  \href{https://ajpsblogging.files.wordpress.com/2015/03/ajps-guide-for-replic-materials-1-0.pdf}{\emph{American
  Journal of Political Science} Guidelines for Preparing Replication
  Files.}
\item
  \href{http://www.stat.columbia.edu/~gelman/research/unpublished/p_hacking.pdf}{Gelman,
  Andrew, and Eric Loken. 2013. ``The Garden of Forking Paths: Why
  Multiple Comparisons Can Be a Problem, Even When There Is No `Fishing
  Expedition' or `\emph{p}-Hacking' and the Research Hypothesis Was
  Posited Ahead of Time.''}
\item
  \href{http://www.pnas.org.proxy.lib.uiowa.edu/content/112/6/1645}{Leek,
  Jeffrey T., and Roger D. Peng. 2015. ``Opinion: Reproducibile Research
  Can Still Be Wrong: Adopting a Prevention Approach.''
  \emph{Proceedings of the National Academy of Sciences}
  112(6):1645-1646} and
  \href{http://biorxiv.org/content/biorxiv/early/2016/07/29/066803.full.pdf}{Patil,
  Prasad, Roger D. Peng, and Jeffrey T. Leek. 2016. ``A Statistical
  Definition for Reproducibility and Replicability.'' \emph{bioRxiv},
  July 29.}
\end{enumerate}

\hypertarget{week-3-january-23-chapter-03}{%
\subsection{Week 3, January 23:
Chapter-03}\label{week-3-january-23-chapter-03}}

\emph{Assignment}:

\begin{enumerate}
\def\labelenumi{\arabic{enumi}.}
\tightlist
\item
  \href{http://happygitwithr.com}{Bryan, Jennifer. 2016.
  \textit{Happy Git and GitHub for the UseR.} Chapters 1-16.} Read this
  carefully and follow its instructions to get set up with \textsf{R},
  RStudio, Git, and GitHub on your laptop before we meet.
\end{enumerate}

\hypertarget{week-4-january-30-chapter-04}{%
\subsection{Week 4, January 30:
Chapter-04}\label{week-4-january-30-chapter-04}}

\emph{Readings}:

\begin{enumerate}
\def\labelenumi{\arabic{enumi}.}
\tightlist
\item
  \href{http://happygitwithr.com}{Bryan, Jennifer. 2016.
  \textit{Happy Git and GitHub for the UseR.} Chapters 1-16.} Read this
  carefully and follow its instructions to get set up with \textsf{R},
  RStudio, Git, and GitHub on your laptop before we meet.
\item
  \href{http://happygitwithr.com}{Bryan, Jennifer. 2016.
  \textit{Happy Git and GitHub for the UseR.} Chapters 1-16.} Read this
  carefully and follow its instructions to get set up with \textsf{R},
  RStudio, Git, and GitHub on your laptop before we meet.
\end{enumerate}

\hypertarget{week-5-february-6-chapter-05}{%
\subsection{Week 5, February 6:
Chapter-05}\label{week-5-february-6-chapter-05}}

\emph{Readings}:

\begin{enumerate}
\def\labelenumi{\arabic{enumi}.}
\tightlist
\item
  \href{http://happygitwithr.com}{Bryan, Jennifer. 2016.
  \textit{Happy Git and GitHub for the UseR.} Chapters 1-16.} Read this
  carefully and follow its instructions to get set up with \textsf{R},
  RStudio, Git, and GitHub on your laptop before we meet.
\end{enumerate}

\hypertarget{week-6-february-13-chapter-06}{%
\subsection{Week 6, February 13:
Chapter-06}\label{week-6-february-13-chapter-06}}

\emph{Readings}: \href{http://happygitwithr.com}{Bryan, Jennifer. 2016.
\textit{Happy Git and GitHub for the UseR.} Chapters 1-16.} Read this
carefully and follow its instructions to get set up with \textsf{R},
RStudio, Git, and GitHub on your laptop before we meet.

\hypertarget{week-7-february-20-chapter-07}{%
\subsection{Week 7, February 20:
Chapter-07}\label{week-7-february-20-chapter-07}}

\emph{Readings}: \href{http://happygitwithr.com}{Bryan, Jennifer. 2016.
\textit{Happy Git and GitHub for the UseR.} Chapters 1-16.} Read this
carefully and follow its instructions to get set up with \textsf{R},
RStudio, Git, and GitHub on your laptop before we meet.

\hypertarget{week-8-february-27-chapter-08}{%
\subsection{Week 8, February 27:
Chapter-08}\label{week-8-february-27-chapter-08}}

\emph{Readings}: \href{http://happygitwithr.com}{Bryan, Jennifer. 2016.
\textit{Happy Git and GitHub for the UseR.} Chapters 1-16.} Read this
carefully and follow its instructions to get set up with \textsf{R},
RStudio, Git, and GitHub on your laptop before we meet.

\hypertarget{week-9-march-6-chapter-09}{%
\subsection{Week 9, March 6:
Chapter-09}\label{week-9-march-6-chapter-09}}

\hypertarget{week-10-march-13-chapter-10}{%
\subsection{Week 10, March 13:
Chapter-10}\label{week-10-march-13-chapter-10}}

\hypertarget{week-11-march-20-chapter-11}{%
\subsection{Week 11, March 20:
Chapter-11}\label{week-11-march-20-chapter-11}}

\hypertarget{week-12-march-27-chapter-12}{%
\subsection{Week 12, March 27:
Chapter-12}\label{week-12-march-27-chapter-12}}

\hypertarget{week-13-april-3-chapter-13}{%
\subsection{Week 13, April 3:
Chapter-13}\label{week-13-april-3-chapter-13}}

\hypertarget{week-14-april-10-chapter-14}{%
\subsection{Week 14, April 10:
Chapter-14}\label{week-14-april-10-chapter-14}}

\hypertarget{week-15-april-17-chapter-15}{%
\subsection{Week 15, April 17:
Chapter-15}\label{week-15-april-17-chapter-15}}

\hypertarget{week-16-april-24-chapter-16}{%
\subsection{Week 16, April 24:
Chapter-16}\label{week-16-april-24-chapter-16}}




\end{document}

\makeatletter
\def\@maketitle{%
  \newpage
%  \null
%  \vskip 2em%
%  \begin{center}%
  \let \footnote \thanks
    {\fontsize{18}{20}\selectfont\raggedright  \setlength{\parindent}{0pt} \@title \par}%
}
%\fi
\makeatother